\documentclass{article}

\usepackage{graphicx} 
\usepackage[french]{babel}
\usepackage[T1]{fontenc}
\usepackage[utf8]{inputenc}
\usepackage{lmodern}
\usepackage{microtype}
\usepackage{hyperref}
\usepackage{amsmath}
\usepackage{amssymb}
\usepackage{geometry}
\usepackage{fancyhdr}
\usepackage{ctex}
\usepackage{tcolorbox}
\pagenumbering{arabic}
\pagestyle{fancy}
\fancyhead[L]{École d'Ingénieurs Paris-SJTU}
\fancyhead[R]{Corentin邱天意}
\fancyfoot[C]{\thepage}
\renewcommand{\headrulewidth}{1pt}
\renewcommand{\footrulewidth}{1pt}

\makeatletter
\@addtoreset{section}{part}
\def\@part[#1]#2{%
    \ifnum \c@secnumdepth >\m@ne
      \refstepcounter{part}%
      \addcontentsline{toc}{part}{\thepart\hspace{1em}#1}%
    \else
      \addcontentsline{toc}{part}{#1}%
    \fi
    {\parindent \z@ \raggedright
     \interlinepenalty \@M
     \normalfont\raggedright
     \ifnum \c@secnumdepth >\m@ne
       \LARGE\bfseries \partname\nobreakspace\thepart
       \par\nobreak
     \fi
     \huge \bfseries #2%
     \markboth{}{}\par}%
    \nobreak
    \vskip 3ex
    \@afterheading}
\renewcommand\partname{Topic}
\makeatother

\title{\textbf{Fiche d'aide du Cours : MATH2307P} \\ Cours assuré par Aurélien KLAK}
\author{Rédigé par Corentin邱天意}
\date{Semestre 2024-2025-2}

\begin{document}

\maketitle

\centerline{\includegraphics[scale=0.4]{sjtu}}

\newpage


\newpage
\section*{Qu'est-ce qu'on va faire?}
\addcontentsline{toc}{part}{Qu'est-ce qu'on va faire?}

Au début, mon idée était de rédiger un sommaire de MATH2307P, l'algèbre linéaire avanvée. Mais après avoir participé au premier cours, j'ai trouvé que le polycopié est bien écrit et que le cours est bien présenté. Cependant, ce que je peux faire, c'est d'écrire un truc supplémentaire pour les autre étudiants. Il y a certainement les personnes qui ont besoin d'une fiche d'aide sur le cours, et le but sera exactement ça.

Qu'est-ce qu'on va mettre ici? J'ai réfiéchi un peu et j'ai décidé que ça pourra devenir un fichier pour les élèves qui ont besoin de ``cram'' avant une khôlle ou un examen. Pour être précis, voici le plan:

\begin{itemize}
 \item Une liste de définitions et de propositions que vous devez connaître, organisée en chapitres.
 \item Leurs démonstrations si Monsieur Aurélien les a fait pandant les cours magistraux.
\end{itemize}



Pour le contenu du cours, les fichiers des TD et le polycopié, visitez la page Moodle de ce cours: \url{http://moodle.speit.sjtu.edu.cn/course/view.php?id=1477}. Si vous avez des retours et des conseils, vous pouvez me trouver avec l'adresse email: \url{Corentin\_TianyiQiu@outlook.com}, ou bien sur Wechat. 


\newpage
\tableofcontents

\newpage

\part{Polynômes}

Le chapitre sur les polynômes est le premier dans ce cours, et sera enseigné pendant les trois premières semaines.


\section{section}


\end{document}